\font\TitleFont=cmr10 scaled 5000
\font\SubTitleFont=cmr10 scaled 2000
\font\SectionFont=cmr10 scaled 1750
\def\PiCa{$\Phi \eta \kappa \alpha$}
\def\newpar{\vskip .25in \noindent}

{\centerline{\TitleFont Pica}}
\vskip .25in
{\centerline{\SubTitleFont A Color Quantizer}}
\vskip .25in
\noindent Pica is a tool that performs color quantization and a number of the
standard tasks associated with this process. Its name comes from the algorithm
it uses to pick the best $\it n$ colors out of a larger set: PCA $\rightarrow$
Principal Component Analysis. The particular implementation of PCA used in Pica
was provided by Dave Moore. The program itself and this documentation was written
by Joel Franklin ({\it jfrankli@reed.edu}).

\newpar
\noindent Given a set of $24$-bit artwork in Microsoft ``.bmp'' format, Pica
will choose up to $236$ colors that best represent the true colors found in the
bitmaps. Then it will optionally remap the bitmaps (with or without dithering) to
this newly created palette and save the resulting data as $8$-bit Microsoft 
{\it .bmp}s.

\newpar
\noindent There are a number of control parameters Pica needs to know about, and these
can be specified in command line format, or in an input file. Either way, the keywords
are the same and below, you will find a detailed list of the $27$ keywords recognized
by the program.

\newpar
{\SectionFont Input File Structure}
\newpar The keywords are divided into three groups: one group controls
the color picking, another sets parameters for the rendering of the bitmaps, 
and a third controls the interpretation of the list of bitmaps to be acted upon.
The keywords themselves can be divided into two classes, functions and flags. Function
keywords take parameters, always given via parentheses and comma-delimited lists.
Flag keywords turn on or off a specific feature. The parser reads in the input file,
and creates a long list of instructions, each line of this list is either a function
keyword, a flag keyword or a filename. The actions performed by Pica are in ``historical''
order down the list. For example, if you are specifying a list of bitmaps to 
be quantized and remapped, and you realize that some number of these should not be
dithered, while the rest should, you can use the flag keyword ``NODITHER'' in
the middle of the filename list, everything after this keyword will not be dithered.
As another example, you may have bitmaps that you want to be considered for quantization,
but do not want to be remapped. In your filename list, you could put these bitmap
names after the flag keyword ``NORENDER''. 

\newpar There will be a number of example input files, covering the range of things
you might want to do with Pica at the end of this document. A keyword list follows.

\newpar {\SectionFont Quantization Keywords}
\newpar {\bf QUANTIZERGB} - tells Pica to quantize in $3$ dimensions (R, G, and B), 
as opposed to $4$ dimensional R, G, B, $\alpha$ (also referred to as RGBA)..

\newpar{\bf QUANTIZEALPHA} - The image data Pica reads will be interpreted as having
an $\alpha$ channel. As of version $1.0$, these $32$-bit formats must be given as
``.tga'' files.

\newpar Only one of these keywords can be specified per call to Pica, because they
indicate such different procedures. It is best (though not necessary) to put one of
these two first in your parse file or command line.

\newpar{\bf CHOOSE($x$,$y$)} - Instructs Pica to choose colors based on the bitmaps in the
file list, and put the chosen colors into a palette between indices $x$ and $y$
(inclusive). To pick the maximal number of colors allowed in Windows, set $x=10$ and
$y=245$.

\newpar{\bf NOCHOOSE} - this is a flag keyword, it turns choosing off for all filenames
below it. So if you have a palette in hand and only want to remap bitmaps, this 
flag should go before any filenames are specified. Alternatively, you can put it in 
the file list, and all files after it will be ignored in choosing optimal colors. 

\newpar{\bf ZEROCOLOR($r$, $g$, $b$)} - Sets the zerocolor (the transparent color) for
a bitmap or set of bitmaps. The color specified is ignored in the bitmap color scan,
and is guaranteed to be mapped to zero in a remap, regardless of dither options. Note that
if you do not set the zero color, and you do not explicitly tell Pica that you do not
want a zero color (see next keyword), the program will assume that the upper left hand
corner of each bitmap is its zero color.

\newpar{\bf ZEROCOLOR} - The flag version of the above, this tells Pica to get the
zero-color from the upper left hand corner of each bitmap.

\newpar{\bf NOZEROCOLOR} - In some cases, you do not want to specify a zero color, nor
do you want the zero color to be pulled in from the upper left corner of the bitmap. This
flag keyword allows you to turn off zeroing altogether. This is useful if you have
tiles, like terrain tiles, in which you don't want peppery transparent speckles. 

\newpar{\bf IGNORECOLORS($r_0$,$g_0$,$b_0$;$\ldots$;$r_n$,$g_n$,$b_n$)} - This function 
keyword allows you to give $n$ colors for the quantizer to ignore. It may be, for
example, that you know that the colors ($15$,$26$,$27$) and ($0$,$0$,$255$) are
over-represented in your art. You would like to run Pica, as usual, but eliminate
these colors' contribution to your palette. This is accomplished via
``IGNORECOLORS($15$,$26$,$27$;$0$,$0$,$255$)''. These colors are remapped to their
closest match in the resulting palette.

\newpar{\bf SHADE($r$,$g$,$b$,$\%$)} - A set of bitmaps run through Pica will be remapped
to the optimal $n$ colors, but you might need to shade and/or haze these bitmaps
in the game itself. Choosing $n$ at the brightest shade and haze levels is unrealistic
at best, and selfish at worst. These bitmaps will rarely appear in the game as
they do in a bitmap viewer. Pica is not in charge of selecting haze and shade colors
for the shade and haze tables that appear in a Phoenix ``.ppl'' file. It is, however,
Pica's job to ensure that the shade and haze tables have decent colors from which to
choose. By specifying a shade color with this keyword, you are telling Pica to take
each pixel of each bitmap and include in the color scan the colors that it runs into
in going from the base pixel color to $\%$ of the shade color, in SHADELEVELS steps (see
next keyword). For example, if you wanted $32$ shade levels in the game, you would
set the SHADELEVELS to $32$, and (since shading is generally done towards black) 
write ``SHADE(0,0,0,80)''. That is, take each color you find, register it with the 
quantizer, but also register the 32 colors you get in going from the original color
to $80\%$ of black. Note that the shade and haze colors that are encountered are registered
with the color scan with the weight of their bitmap. In other words if you specify that
a bitmap should be counted with weight $5$, all of its shade and haze colors will also
be counted with weight $5$.

\newpar{\bf SHADELEVELS($x$)} - Specifies the number of steps to take towards the
desired percentage of the shade color.

\newpar{\bf HAZE($r$,$g$,$b$,$\%$)} - This does the same thing, when used by itself
as the ``SHADE'' keyword. When the two are used together, each shade level color
is taken to $\%$ of the haze color in HAZELEVELS steps. 

\newpar{\bf HAZELEVELS($x$)} - The number of haze steps to take.

\newpar{\bf NOSHADE} - A flag keyword, everything below this will not be shaded. 

\newpar{\bf NOHAZE} - Same as above, but for haze.

\newpar{\bf NOWEIGHT} - Everything below this flag has colors that are counted only once
in the color scan. Normally, Pica increments the value of a color by the weight of the bitmap
(default $1$) every time that color is encountered in the scan. With NOWEIGHT, the value
of each color in the color scan is $1$ regardless of the number of times it appears. The
same is true of the shade and haze colors, these are simply checked for existence in the color
scan, if they exist, their contribution is $1$, if they do not exist, they are given weight
$1$.

\newpar{\bf BLENDRANGE($x$,$y$)} - This is a special keyword that will be covered in more detail
later. Pica can be used to generate translucent colors. That is to say, once you've created
a translucent palette (using QUANTIZEALPHA with targa files), you may want some of the colors
that come up when you blend the translucent colors with a certain range of the game palette
to be represented IN the game palette. Let's say you had an explosion that was orange, and
had some $\alpha$ settings in it. You know that the explosion will always occur on the terrain,
and in your game palette, the terrain colors are $10 \rightarrow 100$. Now in software,
a lookup table is made for every translucent color in the translucent palette blended with
every color in the game palette (using the ``mpal'' tool). But there may not be colors in the
palette that are particularly close to these blended $r$,$g$,$b$ values. Using the
``BLENDRANGE'' keyword you could have, for your game palette, say $10$ entries that are chosen
from all of the $\alpha$-channeled colors in the translucent palette mixed with colors in 
the BLENDRANGE $x \rightarrow y$. This means that when the software lookup table is made using
``mpal'', there are some colors representing the translucent blend in the game palette.


\newpar {\SectionFont Remap Keywords}

\newpar{\bf RENDER($x$,$y$)} - Remaps bitmaps in the range $x \rightarrow y$ of a palette.
The palette must come either from quantization, or from a Microsoft ``.pal'' file. The RENDER
keyword can also be used in incremental palette building. For example, you have a game
palette in which you want to choose $30$ colors for the sky and $100$ for the terrain. You want
to remap the sky and terrain bitmaps in the entire $120$ color palette. This requires three
seperate calls to Pica. In the first, you'll set ``CHOOSE(10,29)''. A palette with these $30$
entries is filled and saved. Now you want to quantize the terrain bitmaps and put their colors
in the appropriate range of the palette ($30\rightarrow 129$), keeping the range $10\rightarrow
29$ fixed. At the same time, you want the quantizer to know about the fixed colors so you don't
get redundancy in the two palette ranges. To do this you would write

RENDER(10,129)


NORENDER


CHOOSE(30,129)

\newpar
Pica reads in the colors from the sky palette which was created first, and registers these
in the color scan with heavy weight (so that they come through the quantizer unchanged, but
considered). Then the program scans in the terrain bitmaps, quantizes everything, and writes
out the palette restoring the original sky colors to their correct location. Finally, a call
to Pica is made that will remap all the bitmaps, sky and terrain, with the render range
set to RENDER(10,129).

You can think of this second use of the RENDER keyword as a virtual render. In picking the
correct $x\rightarrow y$ range for the virtual render, just ask yourself: ``If I was forced
at gunpoint to render a bitmap after this call to Pica, in what range would I do it?''. 

\newpar{\bf NORENDER} - when Pica encounters this keyword, it does not remap any bitmaps
following it.

\newpar{\bf DITHER} - Pica will dither the bitmaps as it remaps them. The dither is a standard
Floyd-Steinberg. As with all flags, this affects the files listed below it. You can turn off
dithering for a set of bitmaps by calling NODITHER (see next keyword). The DITHER range is
the same as the RENDER range. A number of keywords related to dithering follow.

\newpar{\bf NODITHER} - This flag keyword turns off dithering for all files listed below it.

\newpar{\bf RGBDISTANCE} - In dithering or straight remapping, there is the option of 
doing closest match calculations between the true color $24$-bit art and the colors in the
palette in one of three ways. ``RGBDISTANCE'' tells the remapper that you wish to use a 
straight Euclidean distance minimization in finding ``close'' colors. That is
$$
(\hbox{distance})^2 = (r - r_0)^2 + (g - g_0)^2 + (b - b_0)^2
$$

\newpar{\bf WEIGHTEDRGBDISTANCE} - This is the distance scheme used by ``Dynacolor'', it weights
differences in green more heavily than red or blue, because the eye is most sensitive to
differences in green. The formula is
$$
(\hbox{distance})^2 = 3 (r - r_0)^2 + 6 (g - g_0)^2 + (b - b_0)^2
$$

\newpar{\bf LUVDISTANCE} - In this version, Pica does all closest matches using the CIE-LUV
color space. The distance calculation is Euclidean
$$
(\hbox{distance})^2 = (L - L_0)^2 + (U - U_0)^2 + (V - V_0)^2
$$

\newpar{\bf DITHERTOLERANCE($x$)} - When dithering, the original $24$-bit art is perturbed by 
error that diffuses from other pixels. The point is that if there is quite a bit of error
in some color lookups, neighbouring colors can be shifted towards values that they should not
have. A white pixel may acquire a blueish hue. To avoid this, you can set $x$ using the
DITHERTOLERANCE keyword. The value of $x$ is the maximum allowed displacement in each of the
three directions of RGB. For example, if you said ``DITHERTOLERANCE(10)'', then a pixel in the
bitmap with color ($100$,$100$,$100$) is allowed to be changed, through dithering, to 
($110$,$110$,$110$) but if it somehow becomes ($111$,$111$,$111$) (which is greater than
$10$ in each direction), Pica will replace the
offensive color with the original. That is ($111$,$111$,$111$)$\rightarrow$($100$,$100$,$100$).
Note that in order for the tolerance to be set correctly you must specify a lookup distance. 
The default lookup distance is ``RGBDISTANCE'', and if you set the DITHERTOLERANCE before, say, changing
to WEIGHTEDRGBDISTANCE, the tolerance will be setup assuming you are doing RGBDISTANCE. So,
always call this function keyword AFTER setting your lookup type.

\newpar{\SectionFont File Keywords}

\newpar{\bf OUTPATH(pathname)} - Sets the output path. This is where the palette (if you
tell Pica to save one, see next keyword) and all of the remapped $8$-bit art will end up.

\newpar{\bf PALOUTPUTNAME(filename)} - Tells pica to output a palette to the specified
filename, WITH the specified pathname from ``OUTPATH'' prepended. So if you have


OUTPATH(.$\backslash$8bit)


PALOUTPUTNAME(game.pal)

\newpar you would find the palette, when Pica finished, at ``.$\backslash$8bit$\backslash$game.pal''.

\newpar{\bf PALNAME(filename)} - This instructs Pica to read in the palette ``filename'' 
(which must be in Microsoft ``.pal'' format) and use it when quantizing or remapping. The
filename associated with the call to PALNAME is treated as the first of the file list. In
other words, this function keyword must go directly before the bitmap list. This is important,
the flags that affect the entire Pica session (like ``QUANTIZERGB'' or ``QUANTIZEALPHA'') are
referenced through the first entry in the file list. If you call ``PALNAME(filename)'' before
setting the quantization type, the default quantization type will be in place, and will 
be used even if you tell Pica to ``QUANTIZEALPHA'' later on in the input file. The point is:
make your call to PALNAME after you have set up Pica, right before specifying your bitmap
files.

\newpar
The final keyword is more of a convention. When giving Pica bitmap file names, you have the
option of putting parentheses and a number directly after the name, for example
``terrain1.bmp(5)''. The number in the parentheses is the number of times Pica will count the
bitmap. Note that this is different than typing the name ``terrain1.bmp'' into the file list 
five times. Pica scans ``terrain1.bmp(5)'' only once, but adds five to all of the bitmaps colors.
``terrain1.bmp'' typed five times would have the same affect, but it would be scanned five times,
which is time-consuming and redundant. Also note that the weight of the bitmap is the weight
of the shade and haze colors (if you've asked for these) that are registered with the scan. 
Again, this can be accomplished by typing in the filename a number of times, but, for large
numbers of shade and haze levels, you do not want to waste the time with multiple scans.

\newpar
{\SectionFont Examples}

\newpar There are $4$ modes in which Pica is standardly run. I will describe each, and give
sample input files, with descriptions of each line.

\newpar{\bf TYPE 1 - Scan Mode}

\newpar
The first mode for Pica is an RGB scan. Let's say you want to create a game palette with
$236$ colors, and you want to scan bitmaps from the terrain and sky to choose these colors.
You would like to scan these two types of bitmap seperately, and then do a remap using
all of the colors that you've chosen. Your directories are: ``c:$\backslash$game$\backslash$art
$\backslash$sky'' for the $24$-bit
sky bitmaps, ``c:$\backslash$game$\backslash$art$\backslash$terrain'' for the $24$-bit terrain 
bitmaps, and ``c:$\backslash$game$\backslash$art$\backslash$8bit'',
the target 8-bit directory. Your Pica scripts are in the ``c:$\backslash$game$\backslash$art'' directory, and you will
be calling them from that directory. Your first script, the one to generate the sky section
of the palette is called ``sky.pca''. It might look something like this:

QUANTIZERGB


CHOOSE(10,99)


NORENDER 


NOSHADE


NOHAZE


NOZEROCOLOR


OUTPATH(.$\backslash$8bit)


PALOUTPUTNAME(sky.pal)


.$\backslash$sky$\backslash$*.bmp


\newpar Now you have the sky palette set, you want to read it in, quantize the terrain into
a different range, and save off the ``final'' palette. The terrain, you decide should be
shaded and hazed, but you don't want to lose resolution on the terrain tiles, so you use
the default value of $4$ levels of shade and haze. The Pica script, (found in the
same directory) call it ``terrain.pca'', reads:
\vfill\supereject
QUANTIZERGB


RENDER(10,245)


NORENDER


CHOOSE(100,245)


SHADE(0,0,0,80)


HAZE(255,255,255,75)


SHADELEVELS(4)


HAZELEVELS(4)


OUTPATH(.$\backslash$8bit)


PALOUTPUTNAME(game.pal)


PALNAME(.$\backslash$8bit$\backslash$sky.pal)


.$\backslash$terrain$\backslash$*.bmp

\newpar {\bf Type 2 - Remap Mode} 
\newpar You've got your game palette, it's time to remap the art. You want both the sky
and the terrain to use all the colors in the palette, the following script, called 
``c:$\backslash$art$\backslash$remap.pca'' is invoked:

								
NOCHOOSE


RENDER(10,245)


NOZEROCOLOR


OUTPATH(.$\backslash$8bit)


WEIGHTEDRGBDISTANCE


DITHERTOLERANCE(7)


PALNAME($\backslash$8bit$\backslash$game.pal)


.$\backslash$sky$\backslash$*.bmp


.$\backslash$terrain$\backslash$*.bmp


\newpar Let's look closely at these three scripts, and see what we're saying. In the
first one, ``sky.pca'', we tell Pica to ``QUANTIZERGB'', because all of the art in question
is in $24$-bit Microsoft ``.bmp'' format. We want to choose $100$ colors, starting at the
first allowed palette index	$10$. Since this is a call to Pica for choosing colors, we don't 
want to render (because the full palette has not yet been chosen), so the ``NORENDER'' flag
is set. This is the sky, so there's no need to haze or shade (in our hypothetical example). 
We don't want to set a zero color, nor do we want Pica to use the color at the upper left
corner of each bitmap as the zero color, so the flag ``NOZEROCOLOR'' is set. The ``OUTPATH''
is the ``8bit'' directory, we want the output palette to be called ``sky.pal'' and we want
to consider, in our choice of colors, all the $24$-bit art in the ``sky'' directory.

\newpar The second script in this set, ``terrain.pca'' is an example of incremental
palette building. The first line specifies RGB quantization, the second line gives the
range we WOULD render into if we were going to render these bitmaps. It serves to fix
the colors outside the choose range (i.e. the sky colors chosen previously). Then, of course,
we don't want to render (although we could, but only the terrain bitmaps would be rendered,
we'd need a seperate call to Pica to render the sky bitmaps - this is not, strictly speaking
true, we could specify the sky bitmaps under the terrain bitmaps, sandwiching the keyword
``NOCHOOSE'' before them, but this only clutters up the script) so ``NORENDER'' is set. For the
terrain, we've decided to shade to $80\%$ black, and haze towards white (to $75\%$). In 
the two lines after that, we set the shade and haze levels to $4$. Then the familiar 
path setting, with ``OUTPATH'', and the output palette name will be in the ``8bit'' directory,
called ``game.pal''. The key difference between the first script and the second is that in
the second, we use the ``PALNAME'' keyword to pull in the sky palette that has just been
created. Finally, we give the bitmaps from which to choose the colors.

\newpar For the second type of Pica call, we tell Pica ``NOCHOOSE'', this will be a 
remap session. The ``RENDER'' range is given as the entire palette. We don't want a
zero color to be chosen and we don't want to specify one. The ``OUTPATH'' tells Pica
to put all remapped art in the ``8bit'' directory. We will be using a ``WEIGHTEDRGBDISTANCE'',
with ``DITHERTOLERANCE'' set to $7$. Finally, read in the palette ``game.pal'' and give the
bitmap list.

\newpar This is a simple example, we're not weighting bitmaps, there is no set of 
representative bitmaps that we could use to prune down the number of scanned bitmaps, 
etc. These examples should, however, give you a general idea of how to set up your own
Pica scripts.

\newpar{\bf Type 3 - Alpha Quantization}

\newpar Alpha quantization has less options than ``normal'' quantization. You cannot
dither the input data, there is no haze/shade option, and the remap function uses only a 
weighted alpha distance (like 
WEIGHTEDRGBDISTANCE, but with a $(\Delta \alpha)^2$ term thrown in). In general, you
will not be concerned with incrementally building a palette for the translucent bitmaps.
The biggest difference between normal RGB Pica operation and ALPHA quantization is that the
translucent palette can access all $256$ entries, it is not a palette that will
be registered with Windows. The Phoenix libraries use the translucent palette internally, 
so there are no index constraints. 


Continuing with our game example, let's say now that we had a directory for translucent
$24$-bit data, ``c:$\backslash$game$\backslash$translucent'' and we want to remap this data 
to $8$-bit and a palette with an $\alpha$ channel. To do this, we have a script in the 
``c:$\backslash$game$\backslash$art'' directory called ``translucent.pca'' which reads:


QUANTIZEALPHA


CHOOSE(0,255)


NODITHER


NOHAZE


NOSHADE


RENDER(0,255)


OUTPATH(.$\backslash$8bit)


PALOUTPUTNAME(translucent.pal)


.$\backslash$translucent$\backslash$*.tga


\newpar Notice that we are doing everything, palette creation and remapping in one sweep.
That's because only the targas in this directory contribute to the palette, we don't need
to ``wait'' for more colors to be added before we can do a full remap, as we did in the
previous example. The output $8$-bit art has the ``.tga'' extension stripped and replaced
corner color - explosions have transparent edges, we don't want a nice, translucent
explosion with a square black outline. The remapped art is now in the $8$-bit directory
with the rest of the game art. The bitmaps themselves do not have $\alpha$ channels, the
palette that is saved with the bitmap has no $\alpha$ information, that's to make it 
easier to view these with a bitmap viewer. The palette that is output, however, does have
the $\alpha$ channel for each entry.

\newpar{\bf Type 4 - Blended Entries}

\newpar The final task that Pica can perform is to create blended entries in the game
palette. Let's say that in generating the original game palette, we had reserved ten
entries that we want to fill with $\alpha$-blended colors. These are RGB's that result
from taking the $\alpha$ palette (generated above) and blending each color in it with
each color of the game palette. Let's say that we changed the ``terrain.pca'' file so that
it only chose between $100 \rightarrow 235$, and that we went ahead and ran Pica with 
with the ``translucent.pca'' script, but we have yet to run the ``remap.pca'' script. In other
words, we have $10$ colors to fill in, and we want to put in colors that represent
the translucent RGBA colors blended with the terrain portion of the game palette (in our
hypothetical game, explosions only occur on the ground, so there is no dire need for
the explosion RGBA's to be blended with the sky). The following script would do the job,
filling in the range $236\rightarrow 245$ while leaving the rest of the palette unchanged.
This script is called ``c:$\backslash$game$\backslash$art$\backslash$blend.pca'' :


QUANTIZERGB


CHOOSE(236,245)


NODITHER


OUTPATH(.$\backslash$8bit)


RENDER(10,245)


NORENDER


PALOUTPUTNAME(blendedgame.pal)


BLENDRANGE(100,235)


PALNAME(.$\backslash$8bit$\backslash$game.pal)


.$\backslash$8bit$\backslash$translucent.pal


\newpar This is identical to any other scanning script of {\bf Type 1}, except that where the
bitmap list would normally go, there is a translucent palette. The reason for this is that
Pica actually creates a ``bitmap'' out of the colors from the translucent palette and the
indices given by the ``BLENDRANGE'' keyword in the game palette. That is, it scans in colors
with popularity, and fills in the palette in an appropriate manner. Except for the blend
range keyword, this is a straightforward ``bitmap'' scan.

\newpar
{\SectionFont Running Pica}

\newpar Pica can be run using either the command line or a file as input, although in 
general, a file is preferred because it allows you to save your commands, and tweak the
various parameters used. To run Pica with a file, simply type


Pica @filename


\newpar otherwise, you can type ``Pica'' followed by a space-delimeted list of keywords. 


\newpar
{\SectionFont Default Values}

\newpar
It is important to know what the default settings are for the program. It is always best to 
give, explicitly, the options you want for a Pica session, even if they are the default values.
That way, if in a later version of Pica the defaults change, your scripts will still work. It also
makes the goal of your Pica script more obvious to the $\ldots$ casual observer. The following
are set for each file in the file list.



DistanceType      	= RGB


Weight            	= 1


DitherFlag        	= True


QuantizeFlag      	= RGB


RenderFlag        	= False


RenderRangeFirst  	= 10


RenderRangeLast   	= 245


ChooseRangeFirst 	= 0


ChooseRangeLast  	= 0


ShadeTowards      	= NULL


ShadeIntensity    	= 0


HazeTowards       	= NULL


HazeIntensity     	= 0


Filename          	= NULL


OutputPath        	= new char[256] - but not explicitly set!


IgnoreList        	= NULL


ZeroColor         	= NULL


PalOutput         	= NULL


WeightFlag          = True


PalOutput         	= new char[256] - allocated but not filled in!


ChooseFlag       	= True


DitherTolerance    	= 8 - with RGB distance set!


ZeroFlag  		   	= False


BlendFirst		   	= BLENDOFF


BlendLast		   	= BLENDOFF


HazeLevels		   	= 4


ShadeLevels       	= 4


\newpar One final note, that doesn't quite fit anywhere else. The dither function has a built-in
cap on solid--color areas. If a pixel matches {\it exactly} the pixels bordering it, the 
base color is used. Error is calculated, and diffused to neighbouring pixels, but at each
new pixel location, the original (non-perturbed) color is used in checking with its neighbouring
pixels, and again, if these are identical, the original color is brought back. This is to keep
large solid--colored areas looking solid, so there is not a speckly dither pattern across the
bitmap.


\newpar I hope this document has been useful. I can be contacted via e-mail if there are
any questions.


Joel Franklin - July 13th 1997

\end